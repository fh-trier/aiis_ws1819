% Dokumentanklasse: a4paper, 14pt
% Beschreibung:     Dokumentenformat
% Option:           extraarticle - ?
\documentclass[12pt]{article}

% Paket:            setspace
% Beschreibung:     Setz über die optionen den Zeilenabstand
% Optionen:         Möglicher Zeilenabstand
%                   singlespacing = 1,0
%                   onehalfspacing = 1,5
%                   doublespacing = 2,0
% Restriktion:      Muss von Paket hyperref geladen werden. Ansonsten funktioniert das Paket nicht.
\usepackage[onehalfspacing]{setspace}

% Paket:            appendix
% Beschreibung:     Das Paket dient dazu, ausschließlich das Thema einer Überschrift in das Inhaltsverzeichnis zu überführen
% Option:           appendix - Überführt die Überschriften des Anhangs richtig ins das Inhaltsverzeichnis
\usepackage[titletoc]{appendix}

% Paket:            ansmath
% Beschreibung:     Zum darstellen von mathematischen Formeln
\usepackage{amsmath}

% Paket:            bebel
% Beschreibung:     Stellt das Literatur-, Abbilungs-, Tabellenverzeichnis auf eine beliebige Sprache
% Option:           ngerman
\usepackage[ngerman]{babel}

% Paket:            biblatex
% Beschreibung:     Ermöglicht es, ein Literaturverzeichnis zu führen
% Option:
\usepackage[
  style=authoryear-icomp,    % Zitierstil
  isbn=false,                % ISBN nicht anzeigen, gleiches geht mit nahezu allen anderen Feldern
  pagetracker=true,          % ebd. bei wiederholten Angaben (false=ausgeschaltet, page=Seite, spread=Doppelseite, true=automatisch)
  maxbibnames=50,            % maximale Namen, die im Literaturverzeichnis angezeigt werden (ich wollte alle)
  maxcitenames=3,            % maximale Namen, die im Text angezeigt werden, ab 4 wird u.a. nach den ersten Autor angezeigt
  autocite=inline,           % regelt Aussehen für \autocite (inline=\parancite)
  block=space,               % kleiner horizontaler Platz zwischen den Feldern
  backref=true,              % Seiten anzeigen, auf denen die Referenz vorkommt
  backrefstyle=three+,       % fasst Seiten zusammen, z.B. S. 2f, 6ff, 7-10
  date=short,                % Datumsformatbackend=biber
]{biblatex}
\setlength{\bibitemsep}{1em}     % Abstand zwischen den Literaturangaben
\setlength{\bibhang}{2em}        % Einzug nach jeweils erster Zeile
\addbibresource{referenzen//bibliothek.bib}

% Paket:            caption
% Beschreibung:     Bietet eine große Auswahl an Gestaltungsmöglichkeiten bezüglich der Beschriftung von Bildern und Tabellen.
\usepackage{caption}

% Paket:            colortbl
% Beschreibung:     Ermöglicht Tabellen, Spalten oder Zellen farbig zu gestalten.
% Befehle:          \columncolor
% Dokumentation:    http://texdoc.net/texmf-dist/doc/latex/colortbl/colortbl.pdf
\usepackage{colortbl}

% Paket:            subcaption
% Beschreibung:     Bietet eine große Auswahl an Gestaltungsmöglichkeiten bezüglich der Beschriftung von Bildern und Tabellen
%                   die neben oder unter untereinander dargestellt werden sollen.
\usepackage{subcaption}

% format=NAME           - Einstellung des grundsätzlichen Formats von Gleitobjektbeschriftungen
% indention=EINZUG      - Einstellung des Einzugs der Beschriftung ab der zweiten Zeile
% labelformat=NAME      - Einstellung der Zusammensetzung des Bezeichners (z. B. Label und Nummer) der Beschriftung
% labelsep=NAME         - Einstellung des Trenners nach dem Bezeichner
% justification=NAME    - Einstellung er Ausrichtung des Textes Beschriftung
% singlelinecheck=BOOL  - Sonderbehandlung für einzeilige Beschriftungen ein- oder ausschalten
% font=NAME             - Einstellung der Schrift der gesamten Beschriftung
% labelfont=NAME        - Einstellung der Schrift des Bezeichners
% textfont=NAME         - Einstellung der Schrift des Textes der Beschriftung
% margin=BREITE         - Einstellung der Breite des Randes der Beschriftung
% width=BREITE          - Einstellung der Breite der Beschriftung
% parskip=ABSTAND       - Einstellung des Absatzabstandes der Beschriftung
% hangindent=EINZUG     - Einstellung des Einzugs abgesehen von ersten Absatzzeilen
% style=NAME            - Auswahl einer vordefinierten Gesamteinstellung
% aboveskip=ABSTAND     - Einstellung des Abstandes vor einer Beschriftung
% belowskip=ABSTAND     - Einstellung des Abstandes nach einer Beschriftung
% position=WAHL         - Einstellung ob die Beschriftung als Über- oder Unterschrift formatiert werden soll
% tableposition=WAHL    - Einstellung ob nur bei Tabellen die Beschriftung als Über- oder Unterschrift formatiert werden soll
\captionsetup[figure]{position=bottom}
\captionsetup[table]{position=bottom}

% Paket:            courier
% Beschreibung:     Lädt das Paket courier für Schriftarten mit fester Breite.
% Befehle:          \ttfamily     Aktiviert Courier füt Tabellen bzw. generelle begin-Blöcke
\usepackage{courier}

% Paket:            enumitem
% Beschreibung:     Zeilenabstände bei Aufzählungen definieren
% Option:
\usepackage{enumitem}

% Paket:            eurosym
% Beschreibung:     Bildet das Euro-Zeichen in unterschiedlichen Varianten ab
% Option:
\usepackage{eurosym}

% Paket:            fancyhdr
% Beschreibung:     Ermöglich ein generelles Seitenlayout ein zu stellen mit Kopf und Fußzeile.
\usepackage{fancyhdr}

% Paket:            float
% Beschreibung:     Zum Ausrichten von Tabellen und Spalten bzw. deren Zentrierung
% Option:
% Restriktion:      Muss von Paket hyperref geladen werden. Ansonsten funktioniert das Paket nicht.
\usepackage{float}

% Packet:           framemethod
% Beschreibung:
% Option:
\usepackage[framemethod=tikz]{mdframed}
\mdtheorem[
  linecolor=red,
  frametitlefont=\sffamily\bfseries\color{white},
  frametitlebackgroundcolor=red,
]{warn-popup}{Warnung}[subsection]

\mdtheorem[
  linecolor=orange,
  frametitlefont=\sffamily\bfseries\color{white},
  frametitlebackgroundcolor=orange,
]{info-popup}{Information}[subsection]

\mdtheorem[
  linecolor=green,
  frametitlefont=\sffamily\bfseries\color{white},
  frametitlebackgroundcolor=green,
]{example-popup}{Beispiel}

% Paket:            geometry
% Beschreibung:     A4 Seitenabstände
% Option:
\usepackage{geometry}
\geometry{
%  a4paper,           % Papierformat (wird auch über die Dokumentenklasse definiert)
  top=2.7cm,          % Abstand Kopfseite   (Zwischen Kopfseite und Inhalt)
  bottom=2cm,         % Abstand Fußseite    (Zwischen Fußseite und Inhalt)
  left=2.5cm,         % Abstand Linkeseite  (Zwischen Linkerseite und Inhalt)
  right=2cm,          % Abstand Rechteseite (Zwischen Recherseite und Inhalt)
%  width=              %                    textwidth (+marginsep +marinparwidth)
%  textwidth=15cm,     % Textbreite
%  marginparsep=1cm,   % Randnotiztrennng
%  marginparwidth=10cm,% Randnotizbreite
%  height=             %                    textheight (+headheight +headsep + footskip)
%  textheight=        % Texthöhe
  headheight=1cm,     % Kopfhöhe
  headsep=0.5cm,      % Kopftrennung        (Größe zwischen Kopfzeile und Inhalt)
  footskip=1cm,       % Fußzeilenhöhe
}

% Paket:            graphicx
% Beschreibung:     Einbinden von Bildern
% Option:
\usepackage{graphicx}

% Packet:           Hyperref
% Beschreibung:     Importiert hyperref um Querverweise mittels \hyperref zu erzeugen.
% Dokumentation:    https://www.namsu.de/Extra/pakete/Hyperref.html
\usepackage{hyperref}
\hypersetup{
  colorlinks=false,                 % hyperlinks are coloured
  citecolor=green,                  % color for cite links, only visible if colorlinks=true
  linkcolor=red,                    % color for page links, only visible if colorlinks=true
  urlcolor=orange,                  % color for url links, only visible if colorlinks=true
  citebordercolor=green,            % color for citeborder, only visible if colorlinks=true
  urlbordercolor=orange,            % color for url links, only visible if colorlinks=true
  linkbordercolor=red,              % color for page links, only visible if colorlinks=true
  pdfborderstyle={/S/U/W 1},        % border style will be underline of width 1pt
  pdftitle={PDF-Title},
  pdfauthor={Markus Pesch},
  pdfsubject={PDF-Subject},
  pdfkeywords={},
  pdfcreator={pdflatex},
  pdfproducer={LaTeX with hyperref}
}

% Paket:            glossaries
% Beschreibung:     Das Paket glossaries muss nach dem Paket hyperref geladen werden
% Dokumentation:    http://ftp.gwdg.de/pub/ctan/macros/latex/contrib/glossaries/glossaries-user.pdf
% Option/en:
%   acronyms        - This is equivalent to acronym=true and may be used in the document class option list.
%   section         - This is a key=value option. Its value should be the name of a sectional unit (e.g. chapter).
%                     This will make the glossaries appear in the named sectional unit, otherwise each glossary will
%                     appear in a chapter, if chapters exist, otherwise in a section. Unnumbered sectional units will
%                     be used by default.
%   toc             - Add the glossaries to the table of contents.
\usepackage[toc,acronyms]{glossaries}
\makeglossaries
% Akronyme
\newacronym{acr:iana}{IANA}{Internet Assigned Numbers Authority}
\newacronym{acr:sql}{SQL}{Structured-Query-Language}
\newacronym{acr:crud}{CRUD}{Create Read Update Delete}
\newacronym{acr:ddl}{DDL}{Data-Definition-Language}
\newacronym{acr:rest}{REST}{Representational State Transfere}
\newacronym{acr:json}{JSON}{JavaScript Object Notation}
\newacronym{acr:xml}{XML}{Extensible Markup Language}

% Glossar
\newglossaryentry{gls:rest}{
  name={REST},
  description={Representational State Transfer (abgekürzt REST, seltener auch ReST) bezeichnet ein Programmierparadigma für verteilte Systeme, insbesondere für Webservices. REST ist eine Abstraktion der Struktur und des Verhaltens des World Wide Web. REST hat das Ziel, einen Architekturstil zu schaffen, der die Anforderungen des modernen Web besser darstellt. Dabei unterscheidet sich REST vor allem in der Forderung nach einer einheitlichen Schnittstelle (siehe Abschnitt Prinzipien) von anderen Architekturstilen.}
}

\newglossaryentry{gls:json}{
  name={JSON},
  description={Die \acrlong{acr:json}, kurz \acrshort{acr:json}, ist ein kompaktes Datenformat in einer einfach lesbaren Textform zum Zweck des Datenaustauschs zwischen Anwendungen.}
}

\newglossaryentry{gls:ddos}{
  name={DDoS},
  description={Denial of Service bezeichnet in der Informationstechnik die Nichtverfügbarkeit eines Internetdienstes, der eigentlich verfügbar sein sollte. }
}

\newglossaryentry{gls:http-anfragemethoden}{
  name={HTTP-Anfragemethoden},
  description={Im HTTP-Protokoll gibt es verschiedene Anfragemethoden. Sie ermöglichen es dem Browser des Clients Informationen zum Server zu senden, um Dateien abzurufen, ein Formular abzusenden oder eine Datei hochzuladen. }
}

\newglossaryentry{gls:crud}{
  name={CRUD},
  description={Das Akronym \acrshort{acr:crud} umfasst die vier grundlegenden Operationen persistenter Speicher. Create -- Datensatz anlegen, Read oder Retrieve -- Datensatz lesen, Update -- Datensatz aktualisieren und Delete oder Destroy -- Datensatz löschen.}
}

\newglossaryentry{gls:xml}{
  name={XML},
  description={Die \acrlong{acr:xml}, abgekürzt \acrshort{acr:xml}, ist eine Auszeichnungssprache zur Darstellung hierarchisch strukturierter Daten im Format einer Textdatei, die sowohl von Menschen als auch von Maschinen lesbar ist. }
}

\newglossaryentry{gls:rolling-release}{
  name={Rolling Release},
  description={Rolling Release (englisch, aus to roll ‚rollen‘ und release ‚Veröffentlichung‘; sinngemäß „laufende Aktualisierung“) ist ein Begriff aus der Softwaretechnik im Bereich der Betriebssysteme und bedeutet, dass eine kontinuierliche Softwareentwicklung vorliegt. }
}


% Paket:            utf8
% Beschreibung:     Stellt Umlaute richtig dar
% Option:           inputenc - Erlaubt die Darstellung der gleichen Zeichen (Character) wie sie in stdin überliefert werden
\usepackage[utf8]{inputenc}

% Paket:            makeindex
% Beschreibung:     Ermöglicht das Indexieren von Wörter und den Befehl \printindex um den Index auszugeben
\usepackage{makeidx}
\makeindex

% Packet:           Minted
% Beschreibung:     Dient zum highlining von Quellcode wie beispielsweise Java, Bash oder Python.
% Option/en:
%   autogobble:       Leerzeichen zwischen linken Rand und Sourcecode einrücken bzw. weg schneiden.
%   breaklines:       Automatische Zeilenumbrüche
%   cache:            de- oder aktiviert den cache um Sourcecode zwischen zu speichern und so das PDF schneller zu erzeugen
%   cachedir:         Definiert den Pfad zum cache, an dem minted seine Daten zwischen speichern kann
%   fontfamily:       Die Schriftart die benutzt werden soll. tt, courier und helvetica sind vordefiniert.
%   fontsize:         Die Schriftgröße die benutzt werden soll. Beispielsweise fontsize=\footnotesize
%   linenos:          Zeilennummern
%   keywordcase:      Änderung der Buchstaben. Takes lower, upper, or capitalize.
%   showspaces:       Blendet Leerzeichen ein
\usepackage[cache=true]{minted}

% usemintedstyle
% Gebe 'pygmentize -L styles' im Terminal ein um alle verfügbaren styles anzuzeigen.
\usemintedstyle{tango}

% newminted
% Definiere neue aliase um einmalig ein highlighting pro Sprache zu deklarieren
% \newminted{<makroname>}{optionen} ist verfügbar unter "<makroname>code"
\newminted{awk}{autogobble=true, breaklines=true, linenos=true}
\newminted{json}{autogobble=true, breaklines=true, linenos=true}
\newminted{julia}{autogobble=true, breaklines=true, linenos=true}
\newminted{r}{autogobble=true, breaklines=true, linenos=true}
\newminted{sql}{autogobble=true, breaklines=true, linenos=true, keywordcase=upper}
\newminted{xml}{autogobble=true, breaklines=true, linenos=true}

% newmintedfile
% Definiere neue Makros um automatisch Sourcecode aus Dateien zu highlighten.
% \makroname{Dateipfad}
\newmintedfile[inputawk]{awk}{autogobble=true, breaklines=true, linenos=true}
\newmintedfile[inputjson]{json}{autogobble=true, breaklines=true, linenos=true}
\newmintedfile[inputjulia]{julia}{autogobble=true, breaklines=true, linenos=true}
\newmintedfile[inputr]{r}{autogobble=true, breaklines=true, linenos=true}
\newmintedfile[inputsql]{sql}{autogobble=true, breaklines=true, linenos=true, keywordcase=upper}
\newmintedfile{inputxml}{autogobble=true, breaklines=true, linenos=true}

% newmintinline
% Definiere neues Makro um Sourcecoude einzeiler zu highlighten
% \begin{awkcode} \end{awkcode}
\newmintinline{awk}{autogobble=true, breaklines=true, linenos=true}
\newmintinline{json}{autogobble=true, breaklines=true, linenos=true}
\newmintinline{julia}{autogobble=true, breaklines=true, linenos=true}
\newmintinline{r}{autogobble=true, breaklines=true, linenos=true}
\newmintinline{sql}{autogobble=true, breaklines=true, linenos=true, keywordcase=upper}
\newmintinline{xml}{autogobble=true, breaklines=true, linenos=true}

% Paket:            multirow
% Beschreibung:     Zum kombinieren mehrerer Zellen einer Tabelle
% Option:
\usepackage{multirow}

% Paket:            natbib
% Beschreibung:     Für Zitate
% Option:           round - ?
%\usepackage[round]{natbib}

% Paket:            pdflscape
% Beschreibung:     Ermöglicht Seiten horizontal darzustellen
% Option:           \begin{landscape} \end{landscape}
\usepackage{pdflscape}

% Paket:            showframe
% Beschreibung:     Blendet alle Frames (Textkörper, Fußzeile Kopzeile, Seitenrand) ein
% Option:
% \usepackage{showframe}

% Packet:           tabularx
% Beschreibung:     Werden Tabellen mit diesem Paket erstellt, ist es möglich Zeilenumbrüche in einer Zelle zu erzeugen
\usepackage{tabularx}

% Paket:            textpos
% Beschreibung:     Zum erstellen von Textboxen
% \begin{textblock*}{\textwidth}(0cm,0.5cm)
\usepackage{textpos}

% Paket:            tikz
% Beschreibung:     TikZ and PGF are TeX packages for creating graphics programmatically.
% Dokumentation:
\usepackage{tikz}
\usetikzlibrary{intersections}

% Paket:            verbatim
% Beschreibung:     Bildet einen Quelltext exakt ab.
% Options:
\usepackage{verbatim}

% Paket:            warapfig
% Beschreibung:     Ermöglich das floaten von Texten neben Bildern
% Options:
% \begin{wrapfigure}{R}{0.30\textwidth}
%   \includegraphics[width=0.30\textwidth]{img/middleware.png}
%   \caption{Middleware}
% \end{wrapfigure}
\usepackage{wrapfig}

% Packet:           xcolor
% Beschreibung:     Define own color schemas
% Option:
\usepackage{xcolor}

% Definiere Farben
\definecolor{blue}{HTML}{5E7FB8}
\definecolor{brown}{HTML}{BA9D5E}
\definecolor{green}{HTML}{79B960}
\definecolor{grey}{HTML}{7C7C7C}
\definecolor{light-grey}{HTML}{D5D5D5}
\definecolor{orange}{HTML}{FF7F00}
\definecolor{red}{HTML}{DE6144}
\definecolor{violet}{HTML}{B688CB}
\definecolor{white}{HTML}{FFFFFF}
\definecolor{yellow}{HTML}{E2E66C}

% Packet:           csquotes
% Beschreibung:     Muss nach babel, polyglossia, biblatex und inputec geladen werden
% Option:
\usepackage{csquotes}

% Start des Dokuments
\begin{document}

  % Set Globally Table-Margin
  \def\arraystretch{1.2}

  % Fetch Commit ID and Date
  \immediate\write18{./git-info.sh commit > git-id.tmp}
  \immediate\write18{./git-info.sh date > git-date.tmp}
  \immediate\write18{./git-info.sh url > git-url.tmp}

  % Importiere weitere .tex Dokumente
  \begin{titlepage}
  \begin{center}
    \begin{large}
      Flucky Server
    \end{large}

    \begin{huge}
      \begin{singlespace}
            \textbf{Architektur/Implementierung Integrierter Systeme}
      \end{singlespace}
    \end{huge}

    \vspace{0.5cm}

    \begin{figure}[h]
      \centering
      \includegraphics[width=0.85\textwidth]{img//logo.png}
      \label{img:fh-trier-logo}
    \end{figure}

    \vspace{2cm}
    \begin{large}
      \textit{Markus Pesch} \\
      \href{mailto:peschm@hochschule-trier.de}{\textit{peschm@hochschule-trier.de}}
    \end{large}
    \vspace{2cm}

    Version \input{git-id.tmp} vom \input{git-date.tmp}

  \end{center}
\end{titlepage}

  \pagebreak

  % Pagestyle
  % Setze das Seitenlayout auf fancyhdr um Fuß- und Kopfzeilen zu setzen
  \pagestyle{fancy}

  % Löscht alle Kopf- und Fußzeilen des pagestyles fancyhdr
  \fancyhf{}

  % Fuß- und Kopfzeile des Paketes fancyhdr
  % [L] - Linkeseite      [O] - Ungerade Seitenzahlen         [LE,LO] - Linkeseite, Gerade- und Ungerade Seitenanzahlen
  % [C] - Mitte           [E] - Gerade Seitenanzahlen         [CE]    - Seitenmitte, nur gerade Seitenanzahlen
  % [R] - Rechteseite                                         [RO]    - Rechteseite, nur ungerade Seitenanzahlen
  % \fancyhead    Kopfzeile
  % \fancyfoot    Fußzeile
  \fancyhead[L]{\rightmark}
  \fancyhead[R]{\includegraphics[width=4cm]{img/logo.png}}
  \fancyfoot[L]{$<Modul>$ $<Semester>$ - $<Typ>$}
  \fancyfoot[C]{}

  % Pixelstärke der Kopf- und Fußzeilenlinie
  \renewcommand{\headrulewidth}{1pt}
  \renewcommand{\footrulewidth}{1pt}

  % Agenda
  \tableofcontents
  \pagebreak

  % Setze die Seitenbeginn zurück
  \setcounter{page}{1}
  \fancyfoot[R]{Seite \thepage}

  % Importiere weitere .tex Dokumente
  \section{Übung}
\label{sec:uebung_01}

% ##########################################################################
% ############################### Aufgabe 01 ###############################
% ##########################################################################
\subsection{Aufgabe}
\label{sec:uebung_01.aufgabe_01}
Starten Sie das Skript.

\subsubsection*{Lösung}
\label{sec:uebung_01.aufgabe_01.loesung}


% ##########################################################################
% ################################# Minted #################################
% ##########################################################################
\section{Minted}

% ############################### Begin-Block ##############################
\subsection{Begin-Block}

\subsubsection{AWK-Code}
\begin{awkcode}
  if NR > 3 {
    print "0";
  }
\end{awkcode}

\subsubsection{JSON}
\begin{jsoncode}
  {
    "name": "Molecule Man",
    "age": 29,
    "secretIdentity": "Dan Jukes",
    "powers": [
      "Radiation resistance",
      "Turning tiny",
      "Radiation blast"
    ]
  }
\end{jsoncode}

\subsubsection{R-Code}
\begin{rcode}
  print(match(5, c(2,7,5,3))) #  5 is in 3rd place
  print(seq(from=1,to=3,by=.5) %in% 1:3)
\end{rcode}

\subsubsection{SQL-Code}
\begin{sqlcode}
  SELECT *
  FROM tab;
\end{sqlcode}

\subsubsection{XML}
\begin{xmlcode}
  <?xml version="1.0" encoding="UTF-8"?>
  <note>
    <to>Tove</to>
    <from>Jani</from>
    <heading>Reminder</heading>
    <body>Don't forget me this weekend!</body>
  </note>
\end{xmlcode}

% ################################ Input-File ###############################
\subsection{Inputfile}

\subsubsection{AWK-Code}
\inputsql{examples/test.awk}

\subsubsection{JSON}
\inputsql{examples/test.json}

\subsubsection{R-Code}
\inputsql{examples/test.r}

\subsubsection{SQL-Code}
\inputsql{examples/test.sql}

\subsubsection{XML}
\inputsql{examples/test.xml}

% ################################## Inline #################################
\subsection{Inline}

\subsubsection{SQL-Code}
\sqlinline{SELECT * FROM tab;}


% ##########################################################################
% ################################ Info-Boxen ##############################
% ##########################################################################
\section{Info-Boxen}

% ################################### WARN #################################
\begin{warn-popup}
  Warn-Popup
\end{warn-popup}

% ################################### INFO #################################
\begin{info-popup}
  Info-Popup
\end{info-popup}

% ################################# Example ################################
\begin{example-popup}
  Example-Popup
\end{example-popup}


% ##########################################################################
% ################################# Akronyme ###############################
% ##########################################################################
\section{Akronyme}
\acrfull{acr:sql}\\
\acrlong{acr:sql}\\
\acrshort{acr:sql}\\


% ##########################################################################
% ################################ Text-Boxen ##############################
% ##########################################################################
\section{Text-Boxen}
\begin{textblock*}{\textwidth}(4cm,0cm)
  Ich bin eine Textbox
\end{textblock*}


% ##########################################################################
% ############################### Aufzählungen #############################
% ##########################################################################
\section{Aufzählungen}
\begin{itemize}[itemsep=0pt]
  \item Item
  \item Item
  \item Item
\end{itemize}

\begin{enumerate}[itemsep=0pt]
  \item Item
  \item Item
  \item Item
\end{enumerate}

\begin{itemize}[itemsep=0pt]
  \item[A)] Item
  \item[B)] Item
  \item[C)] Item
\end{itemize}

\begin{description}[itemsep=0pt]
  \item[Itempoint 1] \mbox{} \\
  \lipsum[1]
  \bigskip

  \item[Itempoint 2] \mbox{} \\
  \lipsum[1]
  \bigskip

  \item[Itempoint 3] \mbox{} \\
  \lipsum[1]
\end{description}

% ##########################################################################
% ################################# Tabellen ###############################
% ##########################################################################
\section{Tabellen}

% ################################# tabularx ###############################
\begin{table}[H]
  \begin{tabularx}{\textwidth}{X|X|X|X}
    \textbf{FIRSTNAME} & \textbf{LASTNAME} & \textbf{BIRTHDAY} & \textbf{HIREDATE} \\
    \hline\hline
    Maximilian & Arbeitsscheu & 1998-06-21 & 2007-04-18 \\
    Henry & Großkreutz & 1990-09-01 & 2009-02-10 \\
    Leni & Mayer & 1996-10-15 & 2009-02-10 \\
    $[$\dots$]$ & $[$\dots$]$ & $[$\dots$]$ & $[$\dots$]$ \\
  \end{tabularx}
  \caption{tabularx - beispiel\_01}
  \label{tbl:beispiel_01}
\end{table}

% ########################## tabularx - multicolumn ########################
\begin{table}[H]
  \begin{tabularx}{\textwidth}{X|X|X|X}
    \textbf{FIRSTNAME} & \textbf{LASTNAME} & \textbf{BIRTHDAY} & \textbf{HIREDATE} \\
    \hline\hline
    \multicolumn{2}{c|}{\textbf{Multi-column}} & 1998-06-21 & 2007-04-18 \\
    Henry & Großkreutz & \multicolumn{2}{c}{\textbf{Multi-column}} \\
    \multicolumn{3}{c|}{\textbf{Multi-column}} & 2009-02-10 \\
    $[$\dots$]$ & $[$\dots$]$ & $[$\dots$]$ & $[$\dots$]$ \\
  \end{tabularx}
  \caption{tabularx - beispiel\_02}
  \label{tbl:beispiel_02}
\end{table}


% ############################ tabularx - multirow #########################
\begin{table}[H]
  \ttfamily
  \begin{tabularx}{\textwidth}{X|X|X|X}
    \textbf{FIRSTNAME} & \textbf{LASTNAME} & \textbf{BIRTHDAY} & \textbf{HIREDATE} \\
    \hline\hline
    \multirow{2}{*}{\textbf{Multi-row}} & Arbeitsscheu & 1998-06-21 & 2007-04-18 \\
    & Großkreutz & 1990-09-01 & 2009-02-10 \\
    Leni & Mayer & 1996-10-15 & \multirow{2}{*}{\textbf{Multi-row}} \\
    $[$\dots$]$ & $[$\dots$]$ & $[$\dots$]$ & \\
    \hline
  \end{tabularx}
  \caption{tabularx - beispiel\_03}
  \label{tbl:beispiel_03}
\end{table}


% ################################## tabular ###############################
\begin{table}[H]
  \ttfamily
  \begin{tabular}{l|l|l|l}
    \cellcolor{light-grey} \textbf{FIRSTNAME} & \textbf{LASTNAME} & \textbf{BIRTHDAY} & \cellcolor{grey} \textbf{HIREDATE} \\
    \hline\hline
    Maximilian & Arbeitsscheu & \cellcolor{orange}1998-06-21 & 2007-04-18 \\
    Henry & \cellcolor{red} Großkreutz & 1990-09-01 & 2009-02-10 \\
    \cellcolor{green} Leni & Mayer & 1996-10-15 & \cellcolor{blue} 2009-02-10 \\
    $[$\dots$]$ & \cellcolor{yellow} $[$\dots$]$ & \cellcolor{brown} $[$\dots$]$ & $[$\dots$]$ \\
  \end{tabular}
  \caption{tabular - beispiel\_04}
  \label{tbl:beispiel_04}
\end{table}


% ################################## tabular 2 ###############################
\begin{table}[H]
  \begin{tabular}{llll}
    \toprule[2pt]
    \cellcolor{light-grey} \textbf{FIRSTNAME} & \textbf{LASTNAME} & \textbf{BIRTHDAY} & \cellcolor{grey} \textbf{HIREDATE} \\ \midrule[1.5pt]
    Maximilian & Arbeitsscheu & \cellcolor{orange}1998-06-21 & 2007-04-18 \\ \midrule
    Henry & \cellcolor{red} Großkreutz & 1990-09-01 & 2009-02-10 \\ \midrule
    \cellcolor{green} Leni & Mayer & 1996-10-15 & \cellcolor{blue} 2009-02-10 \\ \midrule
    $[$\dots$]$ & \cellcolor{yellow} $[$\dots$]$ & \cellcolor{brown} $[$\dots$]$ & $[$\dots$]$ \\
    \bottomrule
  \end{tabular}
  \caption{tabular - beispiel\_05}
  \label{tbl:beispiel_05}
\end{table}


  % Aufzählung aller Bilder
  % \listoffigures
  % \newpage

  % Aufzählung aller Tabellen
  % \listoftables
  % \newpage

  % Glossary
  % \printglossaries

  % Literatur
  % \printbibliography
\end{document}
