\section{Übung}
\label{sec:uebung_01}

% ##########################################################################
% ############################### Aufgabe 01 ###############################
% ##########################################################################
\subsection{Aufgabe}
\label{sec:uebung_01.aufgabe_01}
Starten Sie das Skript.

\subsubsection*{Lösung}
\label{sec:uebung_01.aufgabe_01.loesung}


% ##########################################################################
% ################################# Minted #################################
% ##########################################################################
\section{Minted}

% ############################### Begin-Block ##############################
\subsection{Begin-Block}

\subsubsection{AWK-Code}
\begin{awkcode}
  if NR > 3 {
    print "0";
  }
\end{awkcode}

\subsubsection{JSON}
\begin{jsoncode}
  {
    "name": "Molecule Man",
    "age": 29,
    "secretIdentity": "Dan Jukes",
    "powers": [
      "Radiation resistance",
      "Turning tiny",
      "Radiation blast"
    ]
  }
\end{jsoncode}

\subsubsection{R-Code}
\begin{rcode}
  print(match(5, c(2,7,5,3))) #  5 is in 3rd place
  print(seq(from=1,to=3,by=.5) %in% 1:3)
\end{rcode}

\subsubsection{SQL-Code}
\begin{sqlcode}
  SELECT *
  FROM tab;
\end{sqlcode}

\subsubsection{XML}
\begin{xmlcode}
  <?xml version="1.0" encoding="UTF-8"?>
  <note>
    <to>Tove</to>
    <from>Jani</from>
    <heading>Reminder</heading>
    <body>Don't forget me this weekend!</body>
  </note>
\end{xmlcode}

% ################################ Input-File ###############################
\subsection{Inputfile}

\subsubsection{AWK-Code}
\inputsql{examples/test.awk}

\subsubsection{JSON}
\inputsql{examples/test.json}

\subsubsection{R-Code}
\inputsql{examples/test.r}

\subsubsection{SQL-Code}
\inputsql{examples/test.sql}

\subsubsection{XML}
\inputsql{examples/test.xml}

% ################################## Inline #################################
\subsection{Inline}

\subsubsection{SQL-Code}
\sqlinline{SELECT * FROM tab;}


% ##########################################################################
% ################################ Info-Boxen ##############################
% ##########################################################################
\section{Info-Boxen}

% ################################### WARN #################################
\begin{warn-popup}
  Warn-Popup
\end{warn-popup}

% ################################### INFO #################################
\begin{info-popup}
  Info-Popup
\end{info-popup}

% ################################# Example ################################
\begin{example-popup}
  Example-Popup
\end{example-popup}


% ##########################################################################
% ################################# Akronyme ###############################
% ##########################################################################
\section{Akronyme}
\acrfull{acr:sql}\\
\acrlong{acr:sql}\\
\acrshort{acr:sql}\\


% ##########################################################################
% ################################ Text-Boxen ##############################
% ##########################################################################
\section{Text-Boxen}
\begin{textblock*}{\textwidth}(4cm,0cm)
  Ich bin eine Textbox
\end{textblock*}


% ##########################################################################
% ############################### Aufzählungen #############################
% ##########################################################################
\section{Aufzählungen}
\begin{itemize}[itemsep=0pt]
  \item Item
  \item Item
  \item Item
\end{itemize}

\begin{enumerate}[itemsep=0pt]
  \item Item
  \item Item
  \item Item
\end{enumerate}

\begin{itemize}[itemsep=0pt]
  \item[A)] Item
  \item[B)] Item
  \item[C)] Item
\end{itemize}

\begin{description}[itemsep=0pt]
  \item[Itempoint 1] \mbox{} \\
  \lipsum[1]
  \bigskip

  \item[Itempoint 2] \mbox{} \\
  \lipsum[1]
  \bigskip

  \item[Itempoint 3] \mbox{} \\
  \lipsum[1]
\end{description}

% ##########################################################################
% ################################# Tabellen ###############################
% ##########################################################################
\section{Tabellen}

% ################################# tabularx ###############################
\begin{table}[H]
  \begin{tabularx}{\textwidth}{X|X|X|X}
    \textbf{FIRSTNAME} & \textbf{LASTNAME} & \textbf{BIRTHDAY} & \textbf{HIREDATE} \\
    \hline\hline
    Maximilian & Arbeitsscheu & 1998-06-21 & 2007-04-18 \\
    Henry & Großkreutz & 1990-09-01 & 2009-02-10 \\
    Leni & Mayer & 1996-10-15 & 2009-02-10 \\
    $[$\dots$]$ & $[$\dots$]$ & $[$\dots$]$ & $[$\dots$]$ \\
  \end{tabularx}
  \caption{tabularx - beispiel\_01}
  \label{tbl:beispiel_01}
\end{table}

% ########################## tabularx - multicolumn ########################
\begin{table}[H]
  \begin{tabularx}{\textwidth}{X|X|X|X}
    \textbf{FIRSTNAME} & \textbf{LASTNAME} & \textbf{BIRTHDAY} & \textbf{HIREDATE} \\
    \hline\hline
    \multicolumn{2}{c|}{\textbf{Multi-column}} & 1998-06-21 & 2007-04-18 \\
    Henry & Großkreutz & \multicolumn{2}{c}{\textbf{Multi-column}} \\
    \multicolumn{3}{c|}{\textbf{Multi-column}} & 2009-02-10 \\
    $[$\dots$]$ & $[$\dots$]$ & $[$\dots$]$ & $[$\dots$]$ \\
  \end{tabularx}
  \caption{tabularx - beispiel\_02}
  \label{tbl:beispiel_02}
\end{table}


% ############################ tabularx - multirow #########################
\begin{table}[H]
  \ttfamily
  \begin{tabularx}{\textwidth}{X|X|X|X}
    \textbf{FIRSTNAME} & \textbf{LASTNAME} & \textbf{BIRTHDAY} & \textbf{HIREDATE} \\
    \hline\hline
    \multirow{2}{*}{\textbf{Multi-row}} & Arbeitsscheu & 1998-06-21 & 2007-04-18 \\
    & Großkreutz & 1990-09-01 & 2009-02-10 \\
    Leni & Mayer & 1996-10-15 & \multirow{2}{*}{\textbf{Multi-row}} \\
    $[$\dots$]$ & $[$\dots$]$ & $[$\dots$]$ & \\
    \hline
  \end{tabularx}
  \caption{tabularx - beispiel\_03}
  \label{tbl:beispiel_03}
\end{table}


% ################################## tabular ###############################
\begin{table}[H]
  \ttfamily
  \begin{tabular}{l|l|l|l}
    \cellcolor{light-grey} \textbf{FIRSTNAME} & \textbf{LASTNAME} & \textbf{BIRTHDAY} & \cellcolor{grey} \textbf{HIREDATE} \\
    \hline\hline
    Maximilian & Arbeitsscheu & \cellcolor{orange}1998-06-21 & 2007-04-18 \\
    Henry & \cellcolor{red} Großkreutz & 1990-09-01 & 2009-02-10 \\
    \cellcolor{green} Leni & Mayer & 1996-10-15 & \cellcolor{blue} 2009-02-10 \\
    $[$\dots$]$ & \cellcolor{yellow} $[$\dots$]$ & \cellcolor{brown} $[$\dots$]$ & $[$\dots$]$ \\
  \end{tabular}
  \caption{tabular - beispiel\_04}
  \label{tbl:beispiel_04}
\end{table}


% ################################## tabular 2 ###############################
\begin{table}[H]
  \begin{tabular}{llll}
    \toprule[2pt]
    \cellcolor{light-grey} \textbf{FIRSTNAME} & \textbf{LASTNAME} & \textbf{BIRTHDAY} & \cellcolor{grey} \textbf{HIREDATE} \\ \midrule[1.5pt]
    Maximilian & Arbeitsscheu & \cellcolor{orange}1998-06-21 & 2007-04-18 \\ \midrule
    Henry & \cellcolor{red} Großkreutz & 1990-09-01 & 2009-02-10 \\ \midrule
    \cellcolor{green} Leni & Mayer & 1996-10-15 & \cellcolor{blue} 2009-02-10 \\ \midrule
    $[$\dots$]$ & \cellcolor{yellow} $[$\dots$]$ & \cellcolor{brown} $[$\dots$]$ & $[$\dots$]$ \\
    \bottomrule
  \end{tabular}
  \caption{tabular - beispiel\_05}
  \label{tbl:beispiel_05}
\end{table}
