% Akronyme
\newacronym{acr:iana}{IANA}{Internet Assigned Numbers Authority}
\newacronym{acr:sql}{SQL}{Structured-Query-Language}
\newacronym{acr:crud}{CRUD}{Create Read Update Delete}
\newacronym{acr:ddl}{DDL}{Data-Definition-Language}
\newacronym{acr:rest}{REST}{Representational State Transfere}
\newacronym{acr:json}{JSON}{JavaScript Object Notation}
\newacronym{acr:xml}{XML}{Extensible Markup Language}

% Glossar
\newglossaryentry{gls:rest}{
  name={REST},
  description={Representational State Transfer (abgekürzt REST, seltener auch ReST) bezeichnet ein Programmierparadigma für verteilte Systeme, insbesondere für Webservices. REST ist eine Abstraktion der Struktur und des Verhaltens des World Wide Web. REST hat das Ziel, einen Architekturstil zu schaffen, der die Anforderungen des modernen Web besser darstellt. Dabei unterscheidet sich REST vor allem in der Forderung nach einer einheitlichen Schnittstelle (siehe Abschnitt Prinzipien) von anderen Architekturstilen.}
}

\newglossaryentry{gls:json}{
  name={JSON},
  description={Die \acrlong{acr:json}, kurz \acrshort{acr:json}, ist ein kompaktes Datenformat in einer einfach lesbaren Textform zum Zweck des Datenaustauschs zwischen Anwendungen.}
}

\newglossaryentry{gls:ddos}{
  name={DDoS},
  description={Denial of Service bezeichnet in der Informationstechnik die Nichtverfügbarkeit eines Internetdienstes, der eigentlich verfügbar sein sollte. }
}

\newglossaryentry{gls:http-anfragemethoden}{
  name={HTTP-Anfragemethoden},
  description={Im HTTP-Protokoll gibt es verschiedene Anfragemethoden. Sie ermöglichen es dem Browser des Clients Informationen zum Server zu senden, um Dateien abzurufen, ein Formular abzusenden oder eine Datei hochzuladen. }
}

\newglossaryentry{gls:crud}{
  name={CRUD},
  description={Das Akronym \acrshort{acr:crud} umfasst die vier grundlegenden Operationen persistenter Speicher. Create -- Datensatz anlegen, Read oder Retrieve -- Datensatz lesen, Update -- Datensatz aktualisieren und Delete oder Destroy -- Datensatz löschen.}
}

\newglossaryentry{gls:xml}{
  name={XML},
  description={Die \acrlong{acr:xml}, abgekürzt \acrshort{acr:xml}, ist eine Auszeichnungssprache zur Darstellung hierarchisch strukturierter Daten im Format einer Textdatei, die sowohl von Menschen als auch von Maschinen lesbar ist. }
}

\newglossaryentry{gls:rolling-release}{
  name={Rolling Release},
  description={Rolling Release (englisch, aus to roll ‚rollen‘ und release ‚Veröffentlichung‘; sinngemäß „laufende Aktualisierung“) ist ein Begriff aus der Softwaretechnik im Bereich der Betriebssysteme und bedeutet, dass eine kontinuierliche Softwareentwicklung vorliegt. }
}
